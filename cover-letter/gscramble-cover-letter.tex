\documentclass[12pt]{letter}
\usepackage{graphicx}
%\usepackage{fontspec}
%\setmainfont{Times New Roman}
%\setmainfont{TeX Gyre Termes}

\textwidth = 6.5 in
\textheight = 9.0 in
\oddsidemargin = 0.0 in
\evensidemargin = 0.0 in
\topmargin = 0.0 in
\headheight = 0.0 in
\headsep = 0.0 in
\parskip = 0.2in
\parindent = 0.0in

\date{}

\address{\mbox{}\vspace*{-1.1in}
\mbox{}\hfill
\includegraphics*{/Users/eriq/Documents/work/nonprj/NMFS_FORMS/NMFS_letterhead_mcallister.pdf}\\}

\signature{Eric C. Anderson \\ \mbox{} \\ eric.anderson@noaa.gov \\ ph: 831-227-4239}

\begin{document}

\begin{letter}{ \today 

Editorial Board \\
Molecular Ecology Resources
}

\pagestyle{empty}
\opening{To the Editor:}

I am delighted to submit the accompanying manuscript,
``{\sc gscramble}: Simulation of admixed individuals without reuse of genetic material,''
to {\em Molecular Ecology Resources}. While it has been understood since the exposition
in Anderson, Waples, \& Kalinowski (2008; {\em Can.\ J.\ Fish.\ Aquat.\ Sci.})\ that care must be
taken when simulating data sets to assess the power of a marker panel for population
assignment and/or genetic stock identification, it has been less widely appreciated that the
same holds true for assessing how much power is available for estimating admixture fractions
and identifying hybrids between closely related groups.  With the advent of data sets
with thousands, or even hundreds of thousands, of genetic markers, this problem
becomes severe.  At the same time, with so many markers, it is critical to account
for physical linkage when assessing how accurately different hybrid categories
can be resolved.

In our paper, we start with several simple simulations to show how dramatic the bias
of power estimation can be with modern data sets, a phenomenon that we refer to
as resampling-induced spurious power inflation, or RISPI.  We then describe a novel,
pedigree-based approach to sampling alleles to create hybrid and admixed individuals
to properly estimate the power for admixture analysis and hybrid identification that avoids
RISPI, and takes full account of physical linkage. We are confident that this approach provides
state-of-the-art methodology for rigorously comparing the outcome of
genetic clustering approaches, such as STRUCTURE, and ADMIXTURE, to results on
simulated data sets that are commensurate with the original data.

In the remainder of the paper, we provide two examples of how the approach can
be used to assess the power for hybrid identification and admixture proportion estimation
in case studies involving Pacific trouts and feral pigs.  

The approach we have developed is implemented in the R package `gscramble' which
is currently available for download from the Comprehensive R Archive
Network (CRAN), and includes full documenation and vignettes.
Members of our team have used `gscramble' already in one published project and
the package is currently in use in several more.   We predict that the package will be
widely used in the future, and as a consequence, the paper will be highly cited.  

Thank you for the opportunity to submit our work to {\em Molecular Ecology Resources}.  


\closing{Sincerely,}
\setlength{\topmargin}{0in}
\end{letter}

 \end{document}   